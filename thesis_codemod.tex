

\chapter{Challenges Of Runtime Code Modification}
\label{sec:ChallengesOfRuntimeCodeModification}
This section discusses various challenges imposed by the CPU and operating
system on runtime code modification. Unless stated otherwise, the discussion 
will focus on the Intel IA-32 architecture and applies to Windows NT kernel mode only. 

Adopting the nomenclature suggested by the Intel processor manuals, 
code writing data to memory with the intent of having the same processor
execute this data as code is referred to as \emph{self-modifying code} \cite{intel07_3A}. 

On SMP machines, it is possible for one processor to write data to memory 
with the intent of having a different processor execute this data as code.
This process if referred to as \emph{cross-modifying code} \cite{intel07_3A}.

For the remainder of this work, the act of executing self-modifying code or
cross-modifying code is also referred to as \emph{runtime code modification}.

\section{Memory Model}
\label{sec:MemoryModel}
In order to implement self-modifying or cross-modifying code, a program must
be able to address the regions of memory containing the code to be modified. Moreover,
due to memory protection mechanisms, overwriting code may not be trivially possible.

The IA-32 architecture offers three memory models -- the flat, segmented and
real mode memory model \cite{intel07_1}. As Windows NT 
relies on the flat memory model, only the flat memory model is examined in this work.

Whenever the CPU fetches code, it addresses memory relative to the segment mapped by
the CS segment register. In the flat memory model, the CS segment register, which refers
to the current code segment, is always set up to map to linear address 0. In the 
same manner, the data and stack segment registers (DS, SS) are set up to refer to 
linear address 0.

It is worth mentioning that AMD64 has retired the use of segmentation \cite{amd64_2} 
and the segment bases for code and data segment are therefore always treated as 0.

Given this setup, code can be accessed and modified on IA-32 as well as on AMD64
in the same manner as data. 

\section{Memory Protection}
One of the features enabled by the use of paging on IA-32 systems is the ability 
to enforce memory protection. Each page can specify restrictions to
which operations are allowed to be performed on memory of the respective page.

In the context of runtime code modification, memory protection is of special 
importance as memory containing code usually does not permit write access,
but rather read and execute access only. A prospective 
solution thus has to provide a means to either circumvent such write protection 
or to temporarily grant write access to the required memory areas.

As other parts of the image are write-protected as well, memory protection 
equally applies to approaches that modify non-code parts of the image such as 
the Import Address Table.

Some instrumentation techniques rely on code generation. Assuming Data Execution Prevention 
\cite{DEP} has been enabled, it is vital for such approaches to work properly that any code
generated is placed into memory regions that grant execute access. While user mode
implementations can rely on a feature of the RTL heap (i.e. using the
\verb|HEAP_CREATE_ENABLE_EXECUTE| when calling RtlCreateHeap) for allocating executable memory, 
no comparable facility for kernel mode exist -- a potential instrumentation solution
thus has to come up with a custom allocation strategy.

\section{Jump distances}
\label{sec:JumpDistances}
A tracing solution may employ dynamic generation of branching instructions in 
order to splice basic blocks and thus to adapt execution flow. For certain approaches,
the offset between the branching instruction itself and the jump target may be of 
significant size. In such cases, the software has to make sure that
the branch instruction chosen does in fact support offsets at least as large as 
required for the individual purpose.

The IA-32 instruction set offers a number of branch instructions which, among other
characteristics, differ in the maximum jump distance they support. As an example, 
a near jump with immediate operand can address targets with an offset relative 
to the current instruction pointer that can be expressed by a signed
32 bit value. Short jumps, in comparison, support only 8 bit offsets. 

However, as those jump instructions that support the largest distances also 
tend to be the largest in size, it is often advantageous to use a smaller, 
yet more restricting jump. In case the instrumentation solution 
is basically rewriting the basic block and its size is allowed to 
change, it may opt to replace a jump instruction by an instruction 
supporting a larger displacement. But in case modifications are performed in-place and the 
size of the basic block is therefore limited, this technique can usually not be applied.

%As this problem is rather common in practice and also applies to jumps from the instrumented
%code into hook routines, a 
A common technique to overcome this issue is to make use of
trampolines \cite{tamches99finegrained}. To support larger jump offsets than the actual
instruction supports, a stopover is introduced: Execution jumps
from the origin to a trampoline, which is a snippet of code consisting of
a single jump instruction only. Taking this jump, execution is then redirected to the
ultimate target. If the trampoline can be located close to the
origin, the injected branch instruction only has to support a very short distance.

On Windows NT, Tamches and Miller \cite{tamches99finegrained} suggest overlaying
initialization code, which can be assumed not to be required any more, yet being close
to the code locations being instrumented, with trampolines.
However, this approach fails to take into consideration that such code regions (i.e.
routines such as \verb|DriverEntry|) are commonly located in paged memory. All accesses
to these memory regions therefore have to occur at \emph{Interrupt Request Level} (IRQL) 
below \verb|DISPATCH_LEVEL| in order not to risk a bugcheck. Yet, as the IRQL at which such 
trampolines might be executed is hard to determine beforehand, adherence to this restriction
may not be guaranteed. As such, the applicability of this approach has to be expected to be limited.

% trampolines: shade, p25

%\section{Non-Intrusiveness}
%By modifying code, the execution behaviour of the traced program is influenced. While
%this influence is intended, it should ideally have no impact on the correctness and
%the outcome of the execution.

\section{Safety Concerns}
\label{sec:SafetyConcerns}
As indicated by the previous section, the act of treating code as data and performing
modifications on code is straightforward. The question remaining to be solved, however, is the
safety of such actions, especially in the presence of multiple processors.

Runtime code modification in general can be used for a variety of purposes and a discussion 
of the overall safety of such actions is out of scope of this work. Therefore,
this section concentrates on the safety in the specific context of using
Runtime Code Patching with the intent of allowing routines to be traced.

Given a multiprogramming environment, an algorithm for dynamic instrumentation using 
runtime code modification shall be considered \emph{safe} in the context of this work if 
its usage guarantees that at any point in time, a routine is either run unchanged, 
%is run in an \emph{augmented} state 
or is run in an \emph{instrumented} state. 



%Running a routine in an augmented state includes:
%\begin{itemize}
%	\item Running an execution path equivalent to the execution path 
%		that would have been taken in case of running the unmodified routine.
%	\item Running any necessary thunking code. 
%\end{itemize}

Running a routine in an instrumented state includes:
\begin{itemize}
	\item Calling a pre-hook routine.
	\item Running an execution path equivalent to the execution path 
		that would have been taken in case of running the unmodified routine.
	\item Calling a post-hook routine. (Optional)
	\item Running any necessary thunking code. 
\end{itemize}

The execution path of an instrumented routine is considered equivalent to the execution
path of an unmodified routine if it includes all non-noop instructions of the unmodified
routine (in the same order) and any additional instructions do not affect the
outcome of the routine in terms of return value, output-parameters and modified memory.
Noop instructions shall include the instruction \verb|nop| as well as any other instructions
that carry out no semantically useful work such as \verb|mov edi, edi|.

The cited criteria should serve as a practical guideline for the evaluation of instrumentation
approaches in the remainder of this work. Both the discussion of whether these criteria are in fact
sufficient for a solution to work correctly under all possible circumstances
and proving whether or not a potential instrumentation algorithm meets these criteria is out of scope
of this work. Rather, the remainder of this section will discuss common issues in
the light of these criteria and their individual ramifications. 

The range of potential issues discussed in the following sections is a summary
of what has been discussed by the publications of existing instrumentation 
solutions presented in section \ref{sec:Classification} as well as issues experienced with existing
tools.

Although not aiming to be a complete list of possible issues, an algorithm 
not being prone to any of these issues while meeting the criteria above should 
be able to be considered \emph{safe for practical usage}.

It is worth noting that this section does not address any issues not applying
to the IA-32 architecture but that may still be relevant on other architectures such as
issues regarding out of order instructions, delay slots or non-transparent instruction caches.

\subsection{Cross-Modifying Code and Atomicity}
\label{sec:Atomicity}
Performing modifications on existing code is a technique commonly encountered among
instrumentation solutions. Assuming a multiprocessor machine, altering code
brings up the challenge of properly synchronizing such activity among processors.

As stated in section \ref{sec:MemoryModel}, code can be modified in the same manner as data. Whether
modifying data is an atomic operation or not, depends on the size of the operand. If the total number of 
bytes to be modified is less than 8 and the target address adheres to certain alignment requirements, 
current IA-32 processors guarantee atomicity of the write operation \cite{intel07_3A}.

If any of these requirements do not hold, multiple write instructions have to be performed, which
is an inherently non-atomic process.
It is, however, crucial to notice that even in situations where using atomic writes 
or bus locking on IA-32 or AMD64 would be feasible, such practice would not necessarily be safe as 
instruction fetches are allowed to pass locked instructions \cite{intel07_3A}.

Although appealing, merely relying on the atomicity of store operations must therefore 
in many cases be assumed to be insufficient for ensuring safe operation.

The exact behavior in case of runtime code modifications slightly varies among
different CPU models. On the one hand, guarantees concerning safety of such practices have been
lessened over the evolvement from the Intel 486 series to the current Core 2 series \cite{intel07_3A}. 
On the other hand, certain steppings of CPU models exhibit defective behavior in this regard \cite{intel02_Err}. 

Due to this variance, the exact range of issues that can arise when performing code modifications
is not clear and appropriate countermeasures cannot be easily identified. As described in several
errata, \cite{intel02_Err}, cross-modifying code not adhering to certain coding practices described later, can
lead to \emph{unexpected execution behavior}, which may include the generation of exceptions. 

The route chosen by the Intel documentation is thus to specify an algorithm that is guaranteed
to work across all processor models -- although for some processors, it might be more restricting than
necessary \cite{intel07_3A}. 

For cross-modifying code, the suggested algorithm makes use of \emph{serializing instructions}.
The role of these instructions, \verb|cpuid| being one of them, is to force any modifications
to registers, memory and flags to be completed and to drain all buffered writes to memory 
before the next instruction is fetched and executed \cite{intel07_3A}. 

Quoting the algorithm defined by Intel in \cite{intel07_3A}:
\begin{quote}
\begin{verbatim}
(* Action of Modifying Processor *)
Memory_Flag <- 0; (* Set Memory_Flag to value other than 1 *)	
Store modified code (as data) into code segment;	
Memory_Flag <- 1;	

(* Action of Executing Processor *)	
WHILE (Memory_Flag != 1)	
  Wait for code to update;	
ELIHW;
	
Execute serializing instruction;	
Begin executing modified code;	
\end{verbatim}
\end{quote}

% performance/cost

To further complicate matters, the IA-32 architecture uses a variable-length instruction set. 
As a consequence of that, additional problems not yet addressed may occur if the instruction lengths of unmodified 
and new instruction do not match. Two situations may occur
\begin{itemize} 
	\item The new instruction is longer than the old instruction. In this case, more than one
			  instruction has to be modified. Modifications straddling instruction boundaries, however,
			  are exposed to an extended set of issues discussed in section \ref{sec:PreemptionAndInterruption}.
	\item The new instruction is shorter than the old instruction. The ramifications of this situation
			  depend on the nature of the new instruction. If, for instance, the instruction is an
			  unconditional branch instruction, the subsequent pad bytes will never be executed and can be neglected. If, 
			  on the other hand, execution may be resumed at the instruction following the new instruction,
			  the pad bytes must constitute valid instructions. For this purpose, a \emph{sled} consisting of
			  \verb|nop| instructions can be used to fill the pad bytes.
			  
			  The algorithm defined by Intel for cross-modifying code ensures that 
			  neither the old nor the new instruction is 
			  currently being executed while the modification is still in progress. Therefore, when employing
			  this algorithm, replacing a single instruction by more than one instruction can 
			  be considered to be equally safe to replacing an instruction by an equally-sized instruction.
\end{itemize}			  

It is worthwhile to notice that regardless which situation applies for instrumentation, the
complementary situation will apply to uninstrumentation.
			  
\subsection{Concurrent Execution}
\label{sec:ConcurrentExecution}
A typical user mode process on a Windows system can be expected to have more than 
one thread. In addition to user threads, the Windows kernel employs a number of 
system threads. Given the presence of multiple threads, it is likely that whenever 
a code modification is performed, more than one thread is affected, 
i.e. more than one thread is sooner or later going to execute the modified
code sequence. 

The basic requirement that has to be met is that even 
in the presence of a preemptive, multi threaded, multiprocessing environment,
an instrumentation solution has to ensure that any other thread either does not run 
the affected code at all, runs code not yet reflecting the respective modifications 
or runs code reflecting the entire set of respective modifications. 

On a multiprocessor system, threads are subject to concurrent execution. While one thread
is currently performing code modifications, another thread, running on a different
processor, may concurrently execute the affected code.

If only a single instruction is to be modified and the cited algorithm for cross-modifying
code is used, concurrent execution, preemption and interruption should not be 
of concern. Any other thread will either execute the old or new instruction, 
but never a mixture of both.

However, the situation is different when more than one instruction is to be modified. In 
this case, a different thread may execute partially modified code.

Although code analysis may indicate certain threads
not to ever call the routine comprising the affected code, signals or \emph{Asynchronous Procedure Calls} (APCs) executed on
this thread may. Therefore, a separation in affected and non-affected threads may not
always be possible and it is safe to assume that all threads are potentially affected.

\subsection{Preemption and Interruption}
\label{sec:PreemptionAndInterruption}
Both on a multiprocessor and a uniprocessor system, all threads running in user mode as 
well as threads running in kernel mode at IRQL APC\_LEVEL or below are subject to 
preemption. Similarly, for a thread running at DISPATCH\_LEVEL or \emph{Device IRQL} (DIRQL), it is also possible 
to be interrupted by a device interrupt \cite{WindowsInternals}. As these situations are similar, only the case
of preemption is discussed.

If only a single instruction is to be modified, preemption and interruption may not
be problematic. If, however, multiple instructions are to be adapted, the ramifications 
of preemption in this context are twofold. On the one hand, the code performing the 
modification may be preempted while being in the midst of a multi-step runtime code 
modification operation:

\begin{itemize}
	\item Thread A performs a runtime code modification. Before the last instruction
		has been fully modified, the thread is preempted. The instruction stream is now in
		a partially-modified state.
	\item Thread B begins executing the code that has been modified by Thread A.
		In case instruction boundaries of old and new code match, the instruction sequence
		that is now run by Thread B should consist of valid instructions only, yet 
		the mixture of old and new code	may define unintended behavior. If instruction 
		lengths do not match, the situation is worse. After Thread B has executed the 
		last fully-modified instruction, the CPU will encounter a partially-overwritten
		instruction. Not being aware of this shift of instruction boundaries, the CPU 
		will interpret the following bytes as instruction stream, which may or may not consist
		of valid instructions. As the code now executed has never been intended to be executed,
		the behavior of Thread B may now be considered arbitrary.
\end{itemize}

%It is worth emphasizing that this situation can occur regardless of whether the underlying
%hardware architecture allows the multiple-instruction modification to be conduced atomically
%or not.

In order to avoid such a situation from occurring, an implementation can disable preemption
and interruption by raising the IRQL above DIRQL during the modification
process. 

On the other hand, the code performing the code modification may run uninterrupted, yet one
of the preempted threads might become affected:

\begin{itemize}
	\item Thread A has begun executing code being part of the instruction sequence that is about
		to be modified. Before having completed executing the last instruction of this sequence,
		it is preempted.
	\item Thread B is scheduled for execution and performs the runtime code modification. Not 
		before all instructions have been fully modified, it is preempted.
	\item Thread A is resumed. Two situations may now occur -- either the memory pointed
		to by the program counter still defines the start of a new instruction or -- due to
		instruction boundaries having moved -- it points into the middle of an instruction. In
		the first case, a mixture of old and new code is executed. In the latter case, the memory
		is reinterpreted as instruction stream. In both cases, the thread is likely to 
		exhibit unintended behavior.
\end{itemize}

One approach of handling such situations is to prevent them from occurring by adapting
the scheduling subsystem of the kernel. However, supporting kernel preemption is a key 
characteristic of the Windows NT scheduler -- removing the ability to preempt kernel 
threads thus hardly seems like an auspicious approach. Regarding the Linux kernel, 
however, it is worth noting that kernel preemption 
is in fact an optional feature supported on more recent versions (2.6.x) only \cite{linux05}. As
a consequence, for older versions or kernels not using this option, the situation as described
in the previous paragraph cannot occur. GILK \cite{Pearce00}, which has been mentioned before
for being an instrumentation tool relying on being able to replace multiple instructions in
fact relies on a kernel not supporting kernel preemption.

A more lightweight approach to this problem relies on analysis of concurrently running
as well as preempted threads. That is, the program counters of all threads are 
inspected for pointing to regions that are about to be affected by the code modification. 
If this is the case, the code modification is deemed unsafe and is aborted. Needless
to say, it is crucial that all threads are either preempted or paused during this analysis 
as well as during the code modification itself. As the thread performing the checks and
modifications is excluded from being paused and analyzed, it has to be assured that this
thread itself is not in danger of interfering with the code modification.

In a similar manner, the return addresses of the stack frames of each
stack can be inspected for pointing to such code regions. Stack walking, however, is exposed
to a separate set of issues that are discussed in section \ref{sec:StackWalking}.

Rather than aborting the operation in case one of the threads is found to be negatively affected by
the pending code modification, a related approach is to attempt to fix the situation. That is,
the program counters of the affected threads are updated so that they can resume properly.

One example for a user-mode solution implementing this approach is 
Detours \cite{Brubacher99}. Before conducting any code modification,
Detours suspends all threads a user has specified as being potentially affected
by this operation. After having completed all code modifications,
all suspended threads are inspected and their program counters are adapted if necessary.
Not before this step has completed, the threads are resumed. 

%However, it is crucial to notice that not only the top stack frame but also any
%other stack frame might become affected by a runtime code modification. Whenever
%the code pointed to by a return address of one of the stack frame is modified,
%the same issues can occur.

%However, Detours
%erroneously assumes that only the current program counter of a thread, but never 
%a return address of one of the thread's stack frames could possibly point to 
%the code under modification. 

 % Fibers --> unknown preemtions!
	  
 % multi-instr-patch: move ip if code is sideeffect-free


\subsection{Basic Block Boundaries}
\label{sec:BasicBlockBoundaries}
Another issue of multiple instruction modification is related to program flow. 
Whenever a sequence of instructions that is to be altered spans multiple 
basic blocks \cite{Aho88}, it is possible that not only the first instruction of the sequence,
but also one of the subsequent instructions may be a branch target.
When instruction boundaries are not preserved by the code modification step,
the branch target might fall into in the midst of one of the new instructions.
Again, such a situation is likely to lead to unintended program behavior 
\cite{tamches99finegrained}.

Identifying basic blocks and thus any potential branch targets requires flow
analysis. However, especially in the case of optimized builds, it is insufficient
to perform an analysis of the affected routine only as blocks might be shared
among more than one routine. In such cases, a routine does not consist of a contiguous
region of code but may be scattered throughout the image. Therefore, it is crucial
to perform flow analysis on the entire image. But even in this case, the existence of 
indirect branches may render a complete analysis impossible in practice. 

The idea of creating a control flow graph in order to avoid overwriting a 
basic block boundary has been implemented by several solutions, including
GILK \cite{Pearce00} and KernInst \cite{Tamches01}, which analyzes the entire kernel image. 
In a similar manner, djprobe \cite{Hiramatsu07} 
performs static code analysis to detect such issues. Both 
instrumentation tools rely on overwriting multiple instructions and are
therefore exposed to this issue.

Another situation where an instrumentation solution may run into the danger of 
overwriting basic block boundaries is the instrumentation of very short routines. 
If the routine is shorter (in terms of instruction bytes occupied) than the
instructions that need to be injected in order to instrument the routine, 
the first basic block(s) of the subsequent routine may be overwritten.


\subsection{Stack Walking}
\label{sec:StackWalking}
As indicated before, walking the stacks of all threads is a technique that can
be employed to check whether certain code modification operations can be expected
to be safe or not. However, for such checks to be reliable, it has to be assured 
that the stack traces are proper, i.e. that no frames are missed.

If correct debugging symbol information is available for all modules involved in a 
stack walk, a stack trace can be expected to be reliable. In practice, however, 
debugging information is often not available, so that symbol-less approaches 
have to be taken.

While reliably obtaining proper stack traces without using debugging symbols may in fact 
be viable on certain architectures and platforms, including Windows NT/AMD64 with its unified calling 
convention, it turns out to be rather problematic on the IA-32 platform. To walk the
stack without debugging symbols, the frame pointer chain can be attempted to be 
traversed. This approach, however, is thwarted as soon as one
of the modules involved makes use of certain compiler optimizations such as 
\emph{frame pointer omission}. As a consequence, this technique can usually
not be expected to yield reliable results.

The dynamic updating solutions Ksplice \cite{Ksplice08} 
and Dynamos therefore implement a more conservative, debugging symbol-less approach of
walking the stack: Starting from the top of the stack, each double word
(on IA-32) is inspected until the base address of the stack has been reached. 
Whenever a double word contains a value that, interpreted as a pointer, refers
to a region in memory occupied by code, this address is assumed to be
a return address of a stack frame. Although this 
assumption may be faulty and is likely to lead to false positives, 
this technique gives these solutions the ability to perform certain 
checks on the routines these addresses point to. 

Although the latter approach seems auspicious, there is another inherent
problem related to the overall approach of analyzing the stacks of all threads.
For such analysis to be reliable, it is of utmost importance to analyze \emph{all}
stacks. In particular, this includes the stacks of all preempted threads. In the normal case,
enumerating this list is possible by walking the list of threads maintained by
the kernel. In user mode Windows NT, such enumeration can be done using the 
\emph{Toolhelp API}, in kernel mode, the list of KTHREAD structures can be
traversed.

However, this approach neglects the fact that in addition to each kernel-allocated
thread being associated a stack, there may be additional, custom allocated stacks.
\emph{Fibers}, for instance, as offered by Windows NT in user mode, offer such a facility. Each fiber 
maintains a separate stack. Yet, as there is no 1:1 mapping between these stacks and
kernel-allocated stacks, it is not possible to enumerate and identify such user-allocated 
stacks by using the thread-enumeration facilities discussed above. In fact, locating 
user-allocated stacks must be expected to not be possible without the individual 
program cooperating. 

In NT kernel mode, the practice of maintaining user-allocated stacks may not be very
common. Moreover, it is explicitly discouraged by Microsoft \cite{Microsoft07}. 
Still, the possibility of drivers using such techniques cannot be ruled out.

Regarding stack walking, custom allocated stacks pose a serious concern.
As they are equally subject to preemption-related issues as kernel-allocated
stacks are, it is crucial for such stacks to be analyzed as well. 


\subsection{Life Cycle Management of Dynamically Allocated Code}
\label{sec:LifecycleManagement}
A common technique encountered among dynamic instrumentation solutions is using
\emph{trampolines}, i.e. little chunks of dynamically generated code
that is mostly specific to a single instrumentation point. Such code blocks, but also
other code that has been dynamically loaded, is subject to proper life cycle management.

As soon as an instrumentation point is revoked and the affected routine is reverted to
its original state, code blocks specific to this instrumentation point are not needed 
any more. However, due to the existence of concurrently running as well as preempted threads, it 
is not trivial to judge whether unloading and freeing the affected memory is indeed a safe operation. For
concurrency, another CPU may currently be running the code regions in question. Even
more likely is the case that one of the preempted threads is referring to instructions
of the affected code chunk. Depending on the individual algorithm and the question whether
or not the code block may contain call instructions, the program counter 
of the preempted thread as well as the return addresses of any of the thread's stack frames
may still point to the code block in question. Unloading and freeing the trampoline 
would in this case inevitably lead to faults or otherwise unintended 
program behavior. 

Detours \cite{Brubacher99} attempts to avoid such situations by analyzing the state of other threads. After
suspending all affected threads -- the distinction between affected and 
non-affected threads has to be made by the user of the library -- Detours checks whether
any thread refers to the trampoline that is to be freed. If this is the case, the
instruction pointer of this thread is modified to point to the original code again, which 
will have been restored before the suspended threads are resumed.

However, by inspecting the current instruction pointer (i.e. the top stack frame) of the threads 
only, Detours fails to consider the possibility of lower stack frames being 
affected: Under certain circumstances, it is possible that Detours will relocate 
a call instruction into the trampoline. In this case, the following situation may occur:
\begin{itemize}
	\item Thread A enters the trampoline, executes the call and enters the called routine.
	\item Thread B removes the detour. This includes freeing the respective trampoline.
	\item Thread A returns from the called routine and will attempt to continue execution
				in the trampoline. As the trampoline has already been freed, an access violation
				or undefined behavior will occur.
\end{itemize}

\emph{MDL} \cite{hollingsworth97mdl}, part of the Paradyn tools, therefore analyzes 
all stack frames before freeing a trampoline. In case an affected stack frame has been
identified, the free operation is delayed and retried later. This analysis is 
significantly more thorough and avoids situations such as the one Detours suffers from.
Yet, the viability of this approach depends on whether it is possible to always 
perform a proper stack walk of the affected threads -- a topic that has been
discussed in section \ref{sec:StackWalking}.

KernInst \cite{tamches99finegrained}, which makes extensive use of dynamically allocated code blocks for trampolines
and code patches, attempts to avoid lifecycle problems from occurring by delaying free operations. 
Solaris, like Windows NT, both supports SMP and kernel thread preemption. Short of 
being able to identify whether a preempted thread or concurrently running thread is still referring
to the affected code, KernInst does not immediately free the memory when the instrumentation
is revoked. Rather, it waits three seconds, during which it expects all affected threads
to have left the critical region before finally freeing the memory. This expectation is backed
by the fact that all operations performed within trampolines are non-blocking.
Yet, this strategy is only able to minimize rather than to eliminate the likeliness
of freeing code blocks that are still in use.

Attempts to solve this issue by using reference counting, as suggested by 
\cite{Tamches01}, are exposed to a hen-egg problem: When entering a dynamically
allocated code block, a counter-incrementing instruction is executed, signaling
that the block is in use. Before leaving the block, the counter is decremented. As soon
as the associated instrumentation has been revoked and the counter drops to zero, the block may be 
freed. The problem of this approach lies in the fact that decrementing the counter and 
taking the jump to leave the block cannot be performed in a single, atomic operation. 
The affected thread may successfully have decremented the counter to zero, 
but is preempted before having been able to take the jump to leave the block. 
Referring to the reference counter, the block is now in a state that allows
it to be freed -- yet, a thread is still referring to it. Again, additionally delaying the free 
operation can help minimizing, but is unable to eliminate this risk.

In a similar manner, dynamically allocated code shared among several or all 
instrumentation points is affected by this discussion. For instance, when the instrumentation
solution has been developed as a loadable driver, it may not always be easily possible to 
determine whether unloading this driver would be a safe operation. Although any
instrumentation may have long been revoked, some preempted (possibly starved) thread may still 
be referring to a routine located within the driver that has been called by 
instrumentation code.

% driver unloadability?



\subsection{Disassembly}
\label{sec:Disassembly}
Certain instrumentation algorithms require partial or even complete disassembly of 
the routine to be instrumented. However, the relative complexity of the IA-32 instruction set,
the fact that instructions are of variable length and the problem that code and 
data can often not be clearly separated from each other makes decoding IA-32 machine 
code nontrivial. Moreover, the existence of indirect jumps can thwart attempts
to perform thorough static analysis on the code. 

Faced with certain problematic code and data sequences, both of the two predominant
disassembly algorithms, \emph{linear sweep} and \emph{recursive traversal} can run into
situations where they silently begin to produce wrong results. Although the results
have been shown to be improvable by hybrid approaches, the general problem remains
in existence \cite{schwarz02disassembly}. 

Regarding compiler-generated code, such problematic code sequences rarely occur. Rather,
such code is often a sign of deliberate code obfuscation \cite{Eilam05}. Still,
in the context of reengineering-sensitive software, which not only malware but also
systems like \emph{Digital Rights Management} (DRM) solutions belong to, such code 
obfuscation is quite popular and should not be neglected in practice.

The nature of a dynamic tracing solution implies that it must be capable
of dealing with unknown code, i.e. code that has never been instrumented before with
the given solution. In order to properly cope with such code, an instrumentation 
solution relying on disassembly and static analysis should therefore put 
emphasis on the quality of disassembly as it may directly influence the stability 
of the system. 

% Definition? Following at least one branch?

% assumption: unknowm code --> possibly even malware, obfusc
% fixed len vs. var-len ISA
% --> anti reversing tricks
% linear sweep vs. recursive traversal etc.

\subsection{Parameter Validation}
Although not strictly related to the practice of instrumentation, an important factor
of the safety of instrumentation is proper usage and validation of parameters. Whenever
an instrumentation solution has intercepted a routine call and wishes to access any
of the parameters passed to the routine, it must be capable of dealing with invalid
parameters. The situation where proper parameter validation is especially crucial
for system stability is when any of the parameters contain values specified by a user
mode program. As this is the case for all system service routines, intercepting these
routines and using the parameters passed requires special care. Practices such
as using a pointer without proper validation or dereferencing an object handle without
type checking can undermine the security and stability of the operating system 
\cite{Skywing06}.

\subsection{Other Events}
Further events an instrumentation solution may not have control over may influence 
proper completion of a runtime code modification. An example of such an event are
DMA operations. A device -- without the instrumentation software being aware of --
might start a data transfer affecting the memory location that is currently affected
by a code modification. Due to concurrent accesses, the result of such an operation may
thus become undefined.

		
%	\item Software/Process External Events:
%		Events the affected process does not have direct influence on. Process external events 
%		that may influence atomicity include actions such as remote
%		creation of a thread that may in turn call the affected routine. If the affected memory
%		region is part of a section shared with at least one other process, it is also possible
%		that another process may read from or write to the affected region. This process may
%		read an inconsistent instruction stream or -- in case of writing -- hamper the code
%		modification. It is worth noting that although Windows NT shares the pages of
%		images among different processes, this sharing is performed with Copy-on-Write semantics.
%		As a consequence, this situation can only arise when a process has explicitly created
%		a section and has copied or generated code into the shared memory area. But even without
%		sections, a different process (assuming appropriate security rights) can use remote VM operations 
%		(ZwReadVirtualMemory, ZwWriteVirtualMemory \cite{Nebbett00}) to access the affected memory and
%		hamper the code modification.
%		
%		In a similar fashion it is possible that the kernel makes user mode callbacks, that
%		may directly or indirectly lead to the affected affected routine being called. By default,
%		only a small number of system-provided routines is affected by such callbacks (such as
%		KiUserExceptionDispatcher \cite{Nebbett00}). Although unlikely in practice, the presence of third-party 
%		drivers may entail the possibility of other user mode routines to be called from
%		the kernel.
%		
%		The exampled provided are non-exhaustive.

%	\item Hardware Induced Events (excluding interrupts):

\section{Sharing of Resources}
The vast majority of tracing solutions can be expected to run parts or even most
of their code in the same virtual address space as the tracee. This holds true for
code editing-based systems but may certainly also apply to other instrumentation 
solutions which utilize different tracing techniques.

As such, the tracing
code and the traced code share a number of resources, including memory, registers,
condition flags, and handles. 

While the tracing code is aware of potential conflicts that may arise from this sharing, the traced code is not.
As such, it is crucial to share these resources in a manner that avoids conflicts and
impacts the traced code to the least amount.

Cmelik \cite{cmelik94shade} defines the following four ways to deal with such sharing
of resources:
\begin{itemize}
	\item	Partition the resource. Part of the resource belongs to one contender, and
				part to another. This, for example, is applicable to heap/pool memory.
	\item Time multiplex the resource. Part of the time the resource belongs to one contender
				and part of the time it belongs to another. Additional processing time and storage 
				are required for swapping. Use cases for this approach include sharing registers 
				between the contenders.
	\item Simulate (virtualize) the resource to the tracee.
	\item Unprotected sharing -- changes made by either effect both. This is generally to be 
				avoided. 
\end{itemize}
 
There are ample resources whose sharing between the traced and tracing party may lead to problems.
However, of special interest are those resources that are modified as part of the instrumentation 
process. Depending on the tracing technique, this may be code, function pointers, or other resources.
Such modifications are conducted to influence the execution of the tracee. Yet, running 
in the same address space, any of such modifications may also directly or indirectly
influence the tracing code, which is generally undesired. 

One of the most prominent issue in this regard is reentrance. For example, a routine like 
\verb|malloc| may have been instrumented and is currently traced. Yet, for tracing to work, 
a memory allocation may have to be performed on behalf of the tracing code. However, calling 
\verb|malloc| in this situation will invoke the tracing code again, which in turn will lead
to endless recursion.

But even if such recursion is avoided, reentrance may become a problem. The implementation
of \verb|malloc| in the previous example may not be reentrant or -- worse yet -- may 
deadlock in case of reentrant calls. Again, the situation depicted may well lead to
a crash or a hang.

Code modification, even if conducted safely, may lead to undesired, resource sharing-related 
effects as well: For example, the instrumentation process may involve injecting \verb|int 3| 
instructions into the the code, with the intent of handling the traps appropriately. However, 
if the instrumented program is being debugged, this practice may well interfere with the debugger's
notion of breakpoints, which may also rely on \verb|int 3| instructions being injected. Clearly,
such effects would be a result of the interrupt being shared among both parties without
an ordered sharing strategy.

Properly dealing with shared resources and potential effects such as reentrance is therefore of utmost
importance for the proper working of a tracing solution. 



\section{Evaluation of Safety Concerns}
Having discussed prevalent challenges of runtime code modification certain conclusions may
be drawn. An important observation is that due to the inherent complexities and challenges
coming along with replacing multiple instructions, an approach that only requires replacing
a single instruction of non-quiescent code is clearly favorable.

Moreover, not all issues are equally critical. Some events such as concurring DMA operations
on memory containing the code to be modified are unlikely and uncommon enough to be neglected
in practice.

	  %UM: threads in same address space + processes sharing pages
	  %COW
	  %KM: all threads
	  %WriteProcMem? kernel grabbing into user space
	  %DMA?
	  
	  
	  




