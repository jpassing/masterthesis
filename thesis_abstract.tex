\chapter*{Abstract\markboth{Abstract}{Abstract}}
Dynamic tracing can be utilized for a variety of purposes, debugging, performance evaluation, 
and program analysis being amongst them. Although the implementations of respective tracing 
systems tend to differ sharply, a limited number of tracing \emph{techniques}
can be identified which all of these solutions base on. Based on this insight, part I of this thesis 
discusses these tracing techniques in detail and proposes an appropriate classification scheme. This scheme
promises to allow both current and future tracing solutions to be classified based on their usage
of these tracing techniques.

In its second part, this thesis discusses NTrace, a dynamic function boundary tracing 
solution for Windows NT kernel mode components that has been developed as part of 
this effort. NTrace not only demonstrates how synergies with Microsoft's Hotpatching 
technology can be utilized in order to achieve safety regarding runtime code modification. 
It also stands out due to deep integration with the exception handling infrastructure of Windows, 
\emph{Structured Exception Handling}. With the ability to trace exception unwinds, NTrace is able 
to yield more precise results than a sheer function entry/exit tracing approach would allow.

By not restricting the usage to customized kernel versions but providing support for retail 
editions of IA-32 Windows NT, NTrace also promises general applicability. Finally, the performance 
of NTrace, and the overhead imposed by tracing activity, is discussed in part III, which 
concludes the thesis.


\chapter*{Zusammenfassung\markboth{Zusammenfassung}{Zusammenfassung}}
Dynamische Ablaufverfolgung kann f�r eine Vielzahl von Zwecken eingesetzt werden, wie etwa
zu Performance-Messungen, Programmanalyse oder zur Fehleranalyse auf Produktivsystemen. 
Obschon die Implementierungen entsprechender Ablaufverfolgungs-Systeme sich bisweilen stark zu 
unterscheiden neigen, so kann doch eine kleine Menge von \emph{Techniken} 
identifiziert werden, auf denen diese basieren. Ausgehend von
dieser Erkenntnis befasst sich Teil I dieser Arbeit mit den Details dieser Techniken und schl�gt
ein Klassifikationsschema f�r diese vor. Das Schema verspricht, sowohl
derzeitige als auch k�nftige Ablaufverfolgungs-Systeme anhand der von ihnen verwendeten Techniken
einordnen zu k�nnen.

Teil II der Arbeit befasst sich mit NTrace, einem Werkzeug zur dynamischen Ablaufverfolgung von 
Funktionsein- und austritten in Windows NT Kern-Komponenten. Es handelt sich hierbei 
um eine Neuentwicklung, die im Rahmen dieser Arbeit entstanden ist. NTrace zeichnet sich 
dabei insbesondere durch zweierlei Aspekte aus: Zum einen wird aufgezeigt, wie Synergien mit der
Microsoft Hotpatching-Technologie genutzt werden k�nnen, um sichere Anwendung von selbstmodifizierendem
Code zu gew�hrleisten. Zum anderen umfasst NTrace eine tiefgreifende Integration mit der
Ausnahmebehandlungs-Infrastruktur des NT Kerns, \emph{Structured Exception Handling}. Diese
Integration erm�glicht NTrace die Ablaufverfolgung von Ausnahmeabwicklungen, um so pr�zisere Ergebnisse
zu erlangen.

NTrace erfordert keinerlei Anpassungen am NT Kern und kann somit auf handels�blichen Editionen
des Windows NT IA-32 Kerns genutzt werden, was wiederum vielf�ltige Anwendungsm�glichkeiten verspricht.
Eine Betrachtung der Leistung von NTrace und dem mit der Ablaufverfolgung verbundenem Overhead
ist Teil von Teil III, welcher diese Arbeit abschlie�t.




