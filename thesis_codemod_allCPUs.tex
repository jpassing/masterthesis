
Having discussed the various instrumentation approaches, a number of cross-cutting
issues and challenges for such solutions can be identified.

usually runs in same address space
-> resource sharing
 -> shade, �3.6:
   - partitioning, time multiplex, simulate, unprotected (-> Floating point)
   
   
   
 - heap/pool
 - handles etc
 
 - reentrance
   - hook may be preemted (ISR, APC, normal scheduling)
   - hook must not call hooked routine

 - self-modifying code

LOCAL PROBES(PER PROCESS) VS GLOBAL PROBES(EXECUTABLE FILE)
mixing solutions - e.g. instr + debugger --> mess





\chapter{Challenges Of Runtime Code Modification}
\label{sec:ChallengesOfRuntimeCodeModification}
This section discusses the hardware-challenges for modifying code at runtime. 
Unless stated otherwise, the discussion applies to kernel mode only.

Adopting the nomenclature suggested by the Intel processor manuals, 
code writing data to memory with the intent of having the same processor
execute this data as code is referred to as \emph{self-modifying code} \cite{intel07_3A}. 

On SMP machines, it is possible for one processor to write data to memory 
with the intent of having a different processor execute this data as code.
This process if referred to as \emph{cross-modifying code} \cite{intel07_3A}.

Certain aspects of runtime code modification are specific to the CPU architecture
targeted. In particular, the provisions reqired to implement self- or cross-modifying 
in a safe manner differ among the different architectures. To account for these 
differences, all three processor families currently supported by Windows
are regarded in the following discussion:
\begin{itemize}
	\item IA-32 and  Intel 64 as an example for a 
		\emph{Complex Instruction Set Computer} (CISC).
	\item IA-64 as an example for the 
		\emph{Very Long Instruction Word} architecture.
\end{itemize}

Although the focus of this work lies on Windows NT, which currently does not
support any RISC processors, certain RISC-specifics will be mentioned.

For the remainder of this work, the act of executing self-modifying code or
cross-modifying code is jointly referred to as runtime code modification.

\section{Accessing Code}
In order to implement self-modifying or cross-modifying code, a program must
be able to address the regions of memory containing the code to be modified. Moreover,
due to memory protection mechanisms, overwriting code may not be trivially possible.

Barring memory protection, IA-64 supports accessing code as data. On IA-32,
however, this is not neccessarily the case -- the IA-32 architecture offers three 
memory models -- the flat, segmented and real mode memory model \cite{intel07_1}. As Windows NT 
relies on the flat memory model, only the flat memory model is examined.

Whenever the CPU fetches code, it addresses memory relative to the segment mapped by
the CS segment register. In the flat memory model, the CS segment register, which refers
to the current code segment, is always set up to map to linear address 0. In the 
same manner, the data and stack segment registers (DS, SS) are set up to refer to 
linear address 0.

Intel 64 has retired the use of segmentation \cite{amd64_2} and the segment 
bases for code and data segment are therefore always treated as 0.

Given this setup, code can be accessed and modified on IA-32 as well as on Intel 64
in the same manner as data. 

\section{Memory Protection}
One of the features enabled by paging, which is used on all three processors by
Windows NT, is the ability to enforce memory protection. Each page can specify restrictions to
which operations are allowed to be performed on memory of the respective page.

In the context of runtime code modification, memory protection is of special 
importance as the memory containing code usually does not permit write access,
but rather read and execute access only. A prospective 
solution thus has to provide a means to either circumvent such write protection 
or to temporarily grant write access to the required memory areas.

As other parts of the image are write-protected as well, memory protection 
equally applies to approaches that modify non-code parts of the image such as 
the Import Address Table.

Some instrumentation techniques also rely on code generation. Assuming Data Execution Prevention 
\cite{DEP} has been enabled, it is vital for such approaches to work properly that any code
generated is placed into memory regions that grant execute access. While user mode
implementations can rely on a feature of the RTL heap (i.e. using the
\verb|HEAP_CREATE_ENABLE_EXECUTE| when calling RtlCreateHeap) for allocating executable memory, 
no comparable facility for kernel mode exist -- a potential instrumentation solution
thus has to come up with a custom allocation strategy.

\section{Jump distances}
\label{sec:JumpDistances}
Instrumentation solutions may employ dynamic generation of branching instructions in 
order to splice basic blocks and thus to adapt execution flow. For certain approaches,
the offset between the branching instruction itself and the jump target may be of 
significant size. In such cases, an instrumentation solution has to make sure that
the branch instruction chosen does in fact support offsets at least as large as 
required for the individual purpose.

IA-64 supports two branch instructions, \emph{Branch} (\verb|br|) and 
\emph{Branch Long} (\verb|brl|). However, the latter is currently not implemented
on contemporary Itanium CPUs \cite{itanium06_3} and may thus be disregarded. For direct,
IP-relative branches, \verb|br| supports displacements up to 16 MB (25 bits). In
the context of a 64 bit architecture, it is thus obvious that this instruction in
itself may be insufficient for certain use cases where the required displacements
exceed this limit.

The situation is similar on RISC CPUs such as SPARC. SPARC defines fixed-length 
instructions and imposes similar limits on the maximum displacements
of branch instructions \cite{SparcV9}. SPARC offers a variety of branch instructions, whose
maximum displacements range between 64 KB and 2 MB. The \verb|call| instruction 
supports displacements up to 1 GB. 

IA-32 and Intel 64 are different in this regard as they depict a variable-length architecture.
A very broad set of branch instructions is offered, which, among other
characteristics, differ significantly in the maximum displacement supported.
As such, the displacement supported by the various incarnations of the \verb|jmp| 
instruction range from as little as 8 bit (-128 to 127 bytes) up to 32 or 64 bit,
respectively, including a segment selector.

However, as those jump instructions that support the largest distances also 
tend to be the largest in size, it is often advantageous to use a smaller, 
yet more restricting jump. In case the instrumentation solution 
is basically rewriting the basic block and its size is allowed to 
change, it may opt to replace a jump instruction by an instruction 
supporting a larger displacement. But in case modifications are performed in-place and the 
size of the basic block is therefore limited, this technique can usually not be applied.

The problem of jump distances therefore exists on all CPU architectures regarded and
can be expected to be rather common in practice. A common technique to overcome 
this issue is thus to make use of trampolines \cite{tamches99finegrained}. 
To support larger jump offsets than the actual
instruction supports, a stopover is introduced: Execution jumps
from the origin to a trampoline, which is a snippet of code consisting of
a single jump instruction only. Taking this jump, execution is then redirected to the
ultimate target. If the trampoline can be located close to the
origin, the injected branch instruction only has to support a very short distance.

On Windows NT, Tamches and Miller \cite{tamches99finegrained} suggest overlaying
initialization code, which can be assumed not to be required any more, with trampolines.
However, this approach fails to take into consideration that such code regions (i.e.
routines such as \verb|DriverEntry|) are commonly located in paged memory. As the
IRQL at which such trampolines could be executed is hard to determine beforehand,
paged trampolines can be assumed to be of little use.

% trampolines: shade, p25

\section{Non-Intrusiveness}
By modifying code, the execution behaviour of the traced program is influenced. While
this influence is intended, it should ideally have no impact on the correctness and
the outcome of the execution.

%Register/stack preservation
%Approach 1: (Proc only) CallConvs
%Pb: private funcs, /LTCG
%Approach 2: Be Pessimistic
%Reserve all registers + stack
%Out-stack?!?
%Approach 3: register liveliness analysis
%(KernInst)
%Req's disasm
%Expensive startup (-> KernInst: 15sec)

% Symbols: Pub vs. priv

\section{Safety Concerns}
\label{sec:SafetyConcerns}
As indicated by the previous section, the act of treating code as data and performing
modifications on code is straightforward. However merely overwriting instructions is not
neccessarily sufficient for having the changes taking effect immediately. Moreover, 
such modifications raise the question of safety, which is in particular the case if 
multiple processors are involved

Runtime code modification in general can be used for a variety of purposes and a discussion 
of the overall safety of such actions is out of scope of this work. Therefore,
this section concentrates on the safety in the specific context of using
Runtime Code Patching for instrumenting routines for function boundary tracing.

Given a multiprocessor environment, an algorithm for dynamic instrumentation using 
runtime code modification shall be considered \emph{safe} in the context of this work if 
its usage guarantees that at any point in time, a routine is either run unchanged, 
%is run in an \emph{augmented} state 
or is run in an \emph{instrumented} state. 



%Running a routine in an augmented state includes:
%\begin{itemize}
%	\item Running an execution path equivalent to the execution path 
%		that would have been taken in case of running the unmodified routine.
%	\item Running any necessary thunking code. 
%\end{itemize}

Running a routine in an instrumented state includes:
\begin{itemize}
	\item Calling a pre-hook routine, then...
	\item Running an execution path equivalent to the execution path 
		that would have been taken in case of running the unmodified routine, and finally...
	\item Calling a post-hook routine. (Optional)
	\item Running any necessary thunking code. 
\end{itemize}

The execution path of an instrumented routine is considered equivalent to the execution
path of an unmodified routine if it includes all non-noop instructions of the unmodified
routine (in the same order) and any additional instructions do not affect the
outcome of the routine in terms of return value, output-parameters and modified memory.
Besides instructions such as \verb|nop| (IA-32), noop instructions in this context 
shall also include any other instructions that carry out no semantically useful work such as 
\verb|mov edi, edi| (IA-32).

The cited criteria should serve as a practical guideline for the evaluation of instrumentation
approaches in the remainder of this work. Both the discussion of whether these criteria are in fact
sufficient for a solution to work correctly under all possible circumstances
and proving whether or not a potential instrumentation algorithm meets these criteria is out of scope
of this work. Rather, the remainder of this section will discuss common issues in
the light of these criteria and their individual ramifications. 

The range of potential issues discussed in the following sections is a summary
of what has been discussed by the publications of existing instrumentation 
solutions presented in section \ref{sec:Classification} as well as issues experienced with existing
tools.

Although not aiming to be a complete list of possible issues, an algorithm 
not being prone to any of these issues while meeting the criteria above should 
be able to considered \emph{safe for practical usage}.

%It is worth noting that the section does not address any issues not applying
%to the IA-32 architecture but relevant on other architectures such as
%issues of out of order instructions, delay slots or non-transparent instruction caches.

\subsection{Synchronization and Coherency}
\label{sec:Atomicity}

Performing modifications on existing code is a technique commonly encountered among
instrumentation solutions. As stated in section \ref{sec:MemoryModel}, all processors
regarded in this chapter allow code to be modified in the same manner as data. Although modifying
code can therefore be assumed to be technically straightforward, such practice
involves a number of risks and may thus lead to unexpected effects.

On the one hand, the memory model implemented by the individual processor determines 
when and in which order the modifications take effect. On the other hand,
most contemporary CPUs, including IA64, recent IA-32 and most RISC processors use
dedicated instruction caches and pipelines for performance reasons. 
While accessing code as data is convenient, this also has the consequence of 
having such modifications take the same route as data modifications. That is, only
the data caches, but not neccessarily the instruction cache, will be involved by 
such memory writes, which in turn on certain CPUs can lead to incoherencies between 
these two kinds of caches.

Finally, performing runtime code modifications on a multiprocessor machine
brings up the challenge of properly synchronizing such activity among processors.

The following subsections will discuss these issues in more detail.

\subsubsection{Synchronization}
The first basic issue a runtime code modification solution has to protect against
is concurrent patching of the same piece of code interfering with each other. Ideally,
a code patching operation could be conducted in an atomic manner so that intereference
is avoided without having to resort to a solution utilizing locks or other synchronization
means. For this to work, the entire code modification would have to be performed by a single,
atomic write. 

Whether this is indeed feasible or not first depends on the instruction set offered by 
the individual CPUs. Although the individual requirements differ among processors, 
memory stores adhering to specific alignment and size restrictions are guaranteed to be 
atomic \cite{intel07_3A} \cite{itanium06_2} \cite{SparcV9}.

On RISC architectures such as the SPARC and IA-64 these requirements are easily adhered
to due to the uniform instruction sizes and the fact that instruction sizes are 
equal to or smaller than the maximum size supported for being atomically written. 
Moreover, instructions can always be assumed to be aligned properly on these architectures.

On variable length instruction set architectures such as IA-32 and Intel 64, however, 
these requirements do not neccessarily hold. As a consequence of their variable length, 
instructions may start at any offset and thus may not be machine word-aligned. Combined with
the fact that instructions may be of significant size, situations can arise where replacing
an instruction with a single memory store is not feasible.

If any of these requirements do not hold, multiple write instructions have to be performed, which
is an inherently non-atomic process.

It is, however, crucial to notice that even in situations where using atomic instructions 
on IA-32 or Intel 64 would be feasible, such practice would not necessarily be safe as 
instruction fetches are allowed to pass locked instructions \cite{intel07_3A}.

As the following sections will point out, more safety-related issues exist when dealing
with runtime code modification. Although appealing, merely relying on the atomicty of store
operations must therefore in many cases be assumed to be insufficient for ensuring safe operation.

\subsubsection{Memory Ordering}
Instrumentation of a routine may comprise multiple steps. As an example, a trampoline 
may need to be generated or updated, followed by a modification on the original routine,
which may include updatating or replacing a branch instruction to point to the trampoline.

In such cases, it is essential for maintaining consistency that the code changes 
take effect in a specific order. Otherwise, if the branch was written before the trampoline code
has been stored, the branch would temporarily point to uninitialized memory. 
If multiple CPUs were involved and code became subject to execution while in such an 
inconsistent state, undefined execution behaviour would occur.

The order in which a program specifies memory loads and stores to be conducted is
referred to as \emph{program order}. On processors such as the Intel 386, this
order is preserved. Contemporary processors, however, implement significantly
weaker memory models. In order to speed up execution, these processors allow
certain memory operations to be conducted out of order \cite{adve96shared}. Such reordering may,
in certain situations like the one depicted before, lead to wrong results or to windows
of inconsistency. To avoid such situations from occuring, the program must
explicitly prohibit certain reorderings to be performed, which can be done
by using memory fences.

Respecting the memory model implemented by the processor is thus crucial in order to 
achieve safe operation. Although both read and store operations are subject to 
potential reordering, only reordering of store operations is of interest in the context
of the example depicted above. However, the memory model and the memory order enforced
by the various CPUs addressed in this chapter differs significantly.

IA-32 and Intel 64 implement a rather strong memory model. In particular, memory stores
are always carried out in program order \cite{itanium06_2} -- this holds true for both
uniprocessor and multiprocessor systems \cite{amd64_2}. For the situation 
depicted above, this means that storing the updated branch target is not conducted
before all stores of writing the trampoline have completed.

SPARC V9 offers a choice between three different memory models, which differ in 
their guarantees they provide: \emph{Total Store Order} (TSO), \emph{Partial Store Order} (PSO), 
and \emph{Relaxed Memory Order} (RMO) \cite{SparcV9}. TSO, the strongest memory model among these
three, guarantees presenvation of the order of store operations. As such, no memory fences
are required. Both RMO, the weakest memory model, and PSO do not provide such guarantees.
That is, to ensure that the second store is not carried out before the first store has
completed, an appropriate memory fence instruction, i.e. a \verb|MEMBAR #StoreStore| 
instruction has to be executed between the two stores.

The memory model implemented by IA-64 also allows stores to be conducted out of order.
This can be prevented either by a memory fence or by specifying the second store to have
\emph{release semantics}: By using the \verb|st.rel| instruction rather than \verb|st|,
the processor is indicated that this instruction must not take effect until all
prior \emph{orderable} instructions, which includes the first store, have taken 
effect \cite{itanium06_2}.

\subsubsection{Instruction Cache/Data Store Incoherencies}
As mentioned before, many modern microprocessors, including contemporary IA-32, SPARC and
IA-64 CPUs use dedicated instruction caches. Whether these instruction caches are kept
coherent with data caches depends on the architecture. Again, IA-32 and Intel 64 are 
more forgiving than other CPUs in this regard and keep instruction and data caches 
coherent -- no manual intervention for flushing the instruction cache is required
\footnote{As a consquence of that, NtFlushInstructionCache is essentially a noop on
these architectures}.

SPARC does not maintain this coherency automatically. To have instruction changes take
effect immediately, SPARC requires the developer to 
issue a \verb|FLUSH| instruction for each modified machine word of instructions 
\cite{SparcV9}. In a similar manner, a \verb|fc.i| instruction is required on IA-64
to flush the respective instructions from the instruction cache \cite{itanium06_3}.

\subsubsection{Pipeline/Instruction Cache Incoherencies}
Another issue that may occur when writing self- or cross-modifying code is the
processor's pipeline to become incoherent with the instruction cache. That is,
although the instruction cache contains updated instructions, the CPU may continue
working with outdated instructions for a while.

According to \cite{SparcV9}, SPARC is not exposed to this problem and flushing the
instruction cache is sufficient to avoid this problem. On IA-64, however, this
incoherency can occur -- whether this is in fact a problem or not depends on the
individual usage scenario. However, to force having the instruction flush take 
effect immediately and to synchronize the instruction cache with the instruction 
fetch stream, issueing a \verb|sync.i| instruction is required.

On IA-32, the exact behavior in case of runtime code modification in general and such
incoherencies in particular slightly varies among different models. 
On the one hand, guarantees concerning safety of such practices have been
lessened over the evolvement from the Intel 486 series to the current Core 2 series \cite{intel07_3A}. 
On the other hand, certain steppings of CPU models have been explicitly documented to exhibit 
defective behavior in this regard \cite{intel02_Err}, \cite{intel02_Err2}. Although 
detailed technical information on this topic is not available, these problems seem
to be possible to emerge when code is being modified that is currently in the state
of execution. 

Due to this variance, the exact range of issues that can arise due to runtime
code modification and cache inconherencies is not clear and appropriate countermeasures 
cannot be easily identified. The route chosen by the Intel documentation is thus to 
specify an algorithm that is guaranteed to work across all processor models -- 
although for some processors, it might be more restricting than necessary \cite{intel07_3A}. 

For cross-modifying code, the suggested algorithm makes use of serializing instructions.
The role of these instructions, \verb|cpuid| being among them, is to force any modifications
to registers, memory and flags to be completed and to drain all buffered writes to memory 
before the next instruction is fetched and executed \cite{intel07_3A}. 


Intel 64 not exposed


\subsubsection{Local/Remote Incoherencies}
 - fences, sync.i



% cahce coherency? NUMA?

% Intel 10.4, 10.6 cache/MESI, 17.28.1


% performance/cost

\subsection{Variable Length Instruction Sets}
The IA-32/Intel 64 architecture uses a variable-length instruction set. As a consequence of that, an additional
complication not yet addressed may occur if the instruction lengths of unmodified 
and new instruction do not match. Two situations may occur
\begin{itemize} 
	\item The new instruction is longer than the old instruction. In this case, more than one
			  instruction has to be modified.
	\item The new instruction is shorter than the old instruction. The ramifications of this situation
			  depend on the nature of the new instruction. If, for instance, the instruction is an
			  unconditional branch instruction, the subsequent pad bytes will never be executed and can be neglected. If, 
			  on the other hand, execution may be resumed at the instruction following the new instruction,
			  the pad bytes must constitute valid instructions. For this purpose, a \emph{sled} consisting of
			  \verb|nop| instructions can be used to fill the pad bytes.
			  
			  The algorithm defined by Intel for cross-modifying code ensures that 
			  neither the old nor the new instruction is 
			  currently being executed while the modification is still in progress. Therefore, when employing
			  this algorithm, replacing a single instruction by more than one instruction can 
			  be considered to be equally safe to replacing an instruction by an equally-sized instruction.
			  
			  %
			  % Intel 64!
			  %
\end{itemize}			  


\subsection{Concurrent Execution}
A typical user mode process on a Windows system can be expected to have more than 
one thread. In addition to user threads, the Windows kernel employs a number of 
system threads. Given the presence of multiple threads, it is likely that whenever 
a code modification is performed, more than one thread is affected by a code modification, 
i.e. more than one thread is sooner or later going to execute the modified
code sequence. 

The basic requirement that has to be met is that even 
in the presence of a preemptive, multithreaded, multiprocessing environment,
an instrumentation solution has to ensure that any other thread either does not run 
the affected code at all, runs code not yet reflecting the respective modifications 
or runs code reflecting the entire set of respective modifications. 

On a multiprocessor system, threads are subject to concurrent execution. While one thread
is currently performing code modifications, another thread, running on a different
processor, may concurrently execute the affected code.

If only a single instruction is to be modified and the cited algorithm for cross-modifying
code is used, concurrent execution, preemption and interruption should not be 
of concern. Any other thread will either execute the old or new instruction, 
but never a mixture of both.

However, the situation is different when more than one instruction is to be modified. In 
this case, a different thread may execute partially modified code.

Although code analysis may indicate certain threads
not to ever call the routine comprising the affected code, signals or APCs executed on
this thread may. Therefore, a separation in affected and non-affected threads may not
always be possible and it is safe to assume that all threads are potentially affected.

\subsection{Preemption and Interruption}
\label{sec:PreemptionAndInterruption}
Both on a multiprocessor and a uniprocessor system, all threads running in user mode as 
well as threads running in kernel mode at IRQL APC\_LEVEL or below are subject to 
preemption. Similarly, for a thread running at DISPATCH\_LEVEL or DIRQL, it is also possible 
to be preempted by a device interrupt \cite{WindowsInternals}. As these situations are similar, only the case
of preemption is discussed.

If only a single instruction is to be modified, preemption and interruption may not
be problematic. If, however, multiple instructions are to be adapted, the ramifications 
of preemption in this context are twofold. On the one hand, the code performing the 
modification may be preempted while being in the midst of a multi-step runtime code 
modification operation:

\begin{itemize}
	\item Thread A performs a runtime code modification. Before the last instruction
		has been fully modified, the thread is preempted. The instruction stream is now in
		a partially-modified state.
	\item Thread B begins executing the code that has been modified by Thread A.
		In case instruction boundaries of old and new code match, the instruction sequence
		that is now run by Thread B should consist of valid instructions only, yet 
		the mixture of old and new code	may define unintended behavior. If instruction 
		lengths do not match, the situation is worse. After Thread B has executed the 
		last fully-modified instruction, the CPU will encounter a partially-overwritten
		instruction. Not being aware of this shift of instruction boundaries, the CPU 
		will interpret the following bytes as instruction stream, which may or may not consist
		of valid instructions. As the code now executed has never been intended to be executed,
		the behavior of Thread B may now be considered arbitrary.
\end{itemize}

%It is worth emphasizing that this situation can occur regardless of whether the underlying
%hardware architecture allows the multiple-instruction modification to be conduced atomically
%or not.

In order to avoid such a situation from occurring, an implementation can disable preemption
and interruption by raising the IRQL above DIRQL during the modification
process. 

On the other hand, the code performing the code modification may run uninterrupted, yet one
of the preempted threads might become affected:

\begin{itemize}
	\item Thread A has begun executing code being part of the instruction sequence that is about
		to be modified. Before having completed executing the last instruction of this sequence,
		it is preempted.
	\item Thread B is scheduled for execution and performs the runtime code modification. Not 
		before all instructions have been fully modified, it is preempted.
	\item Thread A is resumed. Two situations may now occur -- either the memory pointed
		to by the program counter still defines the start of a new instruction or -- due to
		instruction boundaries having moved -- it points into the middle of an instruction. In
		the first case, a mixture of old and new code is executed. In the latter case, the memory
		is reinterpreted as instruction stream. In both cases, the the thread is likely to 
		exhibit unintended behavior.
\end{itemize}

One approach of handling such situations is to prevent them from occurring by adapting
the scheduling subsystem of the kernel. Supporting kernel preemption is a key 
characteristic of the Windows NT scheduler -- removing the ability to preempt kernel 
threads thus hardly seems like an auspicious approach. Regarding the Linux kernel, 
however, it is worth noting that kernel preemption 
is in fact an optional feature supported on more recent versions (2.6.x) only \cite{linux05}. As
a consequence, for older versions or kernels not using this option, the situation as described
in the previous paragraph cannot occur. GILK \cite{Pearce00}, which has been mentioned before
for being an instrumentation tool relying on being able to replace multiple instructions in
fact relies on a kernel not supporting kernel preemption.

A somewhat less invasive approach could be to 
identify and handle such situations after they have occurred. Such a solution would
require inspecting all preempted threads and -- in case of a thread being affected -- taking
appropriate measures such as adapting its program counter. 

However, it is crucial to notice that not only the top stack frame but also any
other stack frame might become affected by a runtime code modification. Whenever
the code pointed to by a return address of one of the stack frame is modified,
the same issues can occur.

One example for a user-mode solution implementing this approach is 
Detours \cite{Brubacher99}. Before conducting any code modification,
Detours suspends all threads a user has specified as being potentially affected
by this operation. After having completed all code modifications,
all suspended threads are inspected and their program counters are adapted if necessary.
Not before this step has completed, all threads are resumed. 

%However, Detours
%erroneously assumes that only the current program counter of a thread, but never 
%a return address of one of the thread's stack frames could possibly point to 
%the code under modification. 

 % Fibers --> unknown preemtions!
	  
 % multi-instr-patch: move ip if code is sideeffect-free


\subsection{Basic Block Boundaries}
\label{sec:BasicBlockBoundaries}
Another issue of multiple instruction modification is related to program flow. 
Whenever a sequence of instructions that is to be altered spans multiple 
basic blocks \cite{Aho88}, it is possible that not only the first instruction of the sequence,
but also one of the subsequent instructions may be a branch target.
When instruction boundaries are not preserved by the code modification step,
the branch target might fall into in the midst of one of the new instructions.
Again, such a situation is likely to lead to unintended program behavior 
\cite{tamches99finegrained}.

Identifying basic blocks and thus any potential branch targets requires flow
analysis. However, especially in the case of optimized builds, it is insufficient
to perform an analysis of the affected routine only as blocks might be shared
among more than one routine. In such cases, a routine does not consist of a contiguous
region of code but may be scattered throughout the image. Therefore, it is crucial
to perform flow analysis on the entire image. But even in this case, the existence of 
indirect branches will render a complete analysis impossible in practice. 

The idea of creating a control flow graph in order to avoid overwriting a 
basic block boundary has been implemented by several solutions, including
GILK \cite{Pearce00} and KernInst \cite{Tamches01}, which analyzes the entire kernel image. 
In a similar manner, djprobe \cite{Hiramatsu07} 
performs static code analysis to detect such issues. Both 
instrumentation tools rely on overwriting multiple instructions and are
therefore exposed to this issue.

Another situation where an instrumentation solution may run into the danger of 
overwriting basic block boundaries is the instrumentation of very short routines. 
If the routine is shorter (in terms of instruction bytes occupied) than the
instructions that need to be injected in order to instrument the routine, 
the first basic blocks(s) of the subsequent routine may be overwritten.



\subsection{Life cycle Management of Dynamically Allocated Code}
\label{sec:LifecycleManagement}
A common technique encountered among dynamic instrumentation solutions is using
\emph{trampolines}, i.e. little chunks of dynamically generated code
that is mostly specific to a single instrumentation point. Such code blocks, but also
other code that has been dynamically loaded is subject to proper life cycle management.

As soon as a instrumentation point is revoked and the affected routine is reverted to
its original state, code blocks specific to this instrumentation point are not needed 
any more. However, due to the existence of concurrently running as well as preempted threads, it 
is not trivial to judge whether unloading and freeing the affected memory is indeed a safe operation. For
concurrency, another CPU may currently be running the code regions in question. Even
more likely is the case that one of the preempted threads is referring to instructions
of the affected code chunk. Depending on the individual algorithm and the question whether
or not the code block may contain call instructions, the program counter 
of the preempted thread as well as the return addresses of any of the thread's stack frames
may still point to the code block in question. Unloading and freeing the trampoline 
would in this case inevitably lead to faults or otherwise unintended 
program behavior. 

Detours attempts to avoid such situations by analyzing the state of other threads. After
suspending all affected threads -- the destinction between affected and 
non-affected threads has to be made by the user of the library -- Detours checks whether
any thread refers to the trampoline that is to be freed. If this is the case, the
instruction pointer of this thread is modified to point to the original code again, which 
will have been restored before the suspended threads are resumed.

However, by inspecting the current instruction pointer (i.e. the top stack frame) of the threads 
only, Detours fails to consider the possibility of lower stack frames being 
affected: Under certain circumstances, it is possible that Detours will relocate 
a call instruction into the trampoline. In this case, the following situation may occur:
\begin{itemize}
	\item Thread A enters the trampoline, executes the call and enters the called routine.
	\item Thread B removes the detour. This includes freeing the respective trampoline.
	\item Thread A returns from the called routine and will attempt to continue execution
				in the trampoline. As the trampoline has already been freed, an access violation
				or undefined behavior will occur.
\end{itemize}

\emph{MDL} \cite{hollingsworth97mdl}, part of the Paradyn tools, therefore analyzes 
all stack frames before freeing a trampoline. In case an affected stack frame has been
identified, the free operation is delayed and retried later. This analysis is 
significantly more thorough and avoids situations such as the one Detours suffers from.
Yet, the viability of this approach depends on whether it is possible to always 
perform a proper stack walk of the affected threads.

While reliably obtaining proper stack traces may in fact be viable on certain 
architectures and platforms, including Windows NT/IA-32e with its unified calling 
convention, it turns out to be rather problematic on the IA-32 platform. To walk the
stack, a symbol-less approach can be taken, in which the frame pointer chain 
is attempted to be traversed. This approach, however, is thwarted as soon as one
of the modules involved makes use of certain compiler optimizations such as 
\emph{frame pointer omission}. In this case, proper stack traces can only
be obtained with the help of debugging symbols. Especially when third-party
modules are involved, however, access to proper debugging symbols may not
be available either.

Obtaining proper stack traces can thus be expected to be hardly attainable
in practice. The dynamic updating solutions Ksplice \cite{Ksplice08} 
and Dynamos accept that obtaining
perfect stack traces is not feasible and implement a brute-force approach of
walking the stack: Starting from the top of the stack, each double word
(on IA-32) is inspected until the base address of the stack has been reached. 
Whenever a double word contains a value that, interpretet as a pointer, refers
to a region in memory occupied by code, this address is assumed to be
a return address of a stack frame. Although this 
assumption may be faulty and is likely to lead to false positives, 
this technique gives these solutions the abilitiy to perform certain 
checks on the routines these addresses point to. 

KernInst, which makes extensive use of dynamically allocated code blocks for trampolines
and code patches, attepts to avoid such situations from occuring by delaying free operations. 
Solaris, like Windows NT, both supports SMP and kernel thread preemtion. Short of 
being able to identify whether a preempted thread or concurrently running thread is still referring
to the affected code, KernInst does not immediately free the memory when the instrumentation
is revoked. Rather, it waits three seconds, during which it expects all affected threads
to have left the critical region before finally freeing the memory. This expectation is backed
by the fact that all operations performed within spring boards and code patches are non-blocking.
Yet, this strategy is only able to minimize rather than to eliminate the likeliness
of freeing code blocks that are still in use.

Attempts to solve this issue by using reference counting, as suggested by 
\cite{Tamches01}, are exposed to a hen-egg problem: When entering a dynamically
allocated code block, a counter-incrementing instruction is executed, signalling
that the block is in use. Before leaving the block, the counter is decremented. As soon
as the associated instrumentation has been revoked and the counter drops to zero, the block may be 
freed. The problem of this approach lies in the fact that decrementing the counter and 
taking the jump to leave the block cannot be performed in a single, atomic operation. 
The affected thread may successfully have decremented the counter to zero, 
but is preempted before having been able to take the jump to leave the block. 
Referring to the reference counter, the block is now in a state that allows
it to be freed -- yet, a thread is still referring to it. Again, delaying the free 
operation can only minimize, but not eliminate this risk.

In a similar manner, dynamically allocated code shared among several or all 
instrumentation points is affected by this discussion. Therefore, when the instrumentation
solution has been developed as a loadable driver, it may not always be easily possible to 
determine whether unloading this driver would be a safe operation. Although any
instrumentation may have long been revoked, some preempted (possibly starved) thread may still 
be referring to a routine located within the driver that has been called by 
instrumentation code.

% driver unloadability?



\subsection{Disassembly}
\label{sec:Disassembly}
Certain instrumentation algorithms require a partial or even complete disassembly of 
the routine to be instrumented. However, the relative complexity of the IA-32 instruction set,
the fact that instructions are of variable length and the problem that code and 
data can often not be clearly separated from each other makes decoding IA-32 machine 
code nontrivial. Moreover, the existence of indirect jumps can thwart attempts
to perform thorough static analysis on the code. 

Faced with certain problematic code and data sequences, both of the two predominant
disassembly algorithms, \emph{linear sweep} and \emph{recursive traversal} can run into
situations where they silently begin to produce wrong results. Although the results
have been shown to be improvable by hybrid approaches, the general problem remains
in existence \cite{schwarz02disassembly}. 

Regarding compiler-generated code, such problematic code sequences rarely occur. Rather,
such code is often a sign of deliberate code obfuscation \cite{Eilam05}. Still,
in the context of reenginieering-sensitive software, which not only malware but also
systems like \emph{Digital Rights Management} (DRM) solutions belong to, such code 
obfuscation is quite popular and should not be neglected in practice.

The nature of a dynamic tracing solution implies that it must be capable
of dealing with unknown code, i.e. code that has never been instrumented before with
the given solution. In order to properly cope with such code, an instrumentation 
solution relying on disassembly and static analysis should therefore put 
emphasis on the quality of disassembly as it may directly influence the stability 
of the system. 

% Definition? Following at least one branch?

% assumption: unknowm code --> possibly even malware, obfusc
% fixed len vs. var-len ISA
% --> anti reversing tricks
% linear sweep vs. recursive traversal etc.

\subsection{Parameter Validation}
Although not strictly related to the practice of instrumentation, an important factor
of the safety of instrumentation is proper usage and validation of parameters. Whenever
an instrumentation solution has intercepted a routine call and wishes to access any
of the parameters passed to the routine, it must be capable of dealing with invalid
parameters. The situation where proper parameter validation is especially crucial
for system stability is when any of the parameters contain values specified by a user
mode program. As this is the case for all system service routines, intercepting these
routines and using the parameters passed requires special care. Practices such
as using a pointer without proper validation or dereferencing an object handle without
type checking can undermine the security and stability of the operating system 
\cite{Skywing06}.

\subsection{Other Events}
Further events an instrumentation solution may not have control of may influence 
proper completion of a runtime code modification. An example of such an event are
DMA operations. A device -- without the instrumentation software being aware of --
might start a data transfer affecting the memory location that is currently affected
by a code modification. Due to concurrent accesses, the result of such an operation may
thus become undefined.

		
%	\item Software/Process External Events:
%		Events the affected process does not have direct influence on. Process external events 
%		that may influence atomicity include actions such as remote
%		creation of a thread that may in turn call the affected routine. If the affected memory
%		region is part of a section shared with at least one other process, it is also possible
%		that another process may read from or write to the affected region. This process may
%		read an inconsistent instruction stream or -- in case of writing -- hamper the code
%		modification. It is worth noting that although Windows NT shares the pages of
%		images among different processes, this sharing is performed with Copy-on-Write semantics.
%		As a consequence, this situation can only arise when a process has explicitly created
%		a section and has copied or generated code into the shared memory area. But even without
%		sections, a different process (assuming appropriate security rights) can use remote VM operations 
%		(ZwReadVirtualMemory, ZwWriteVirtualMemory \cite{Nebbett00}) to access the affected memory and
%		hamper the code modification.
%		
%		In a similar fashion it is possible that the kernel makes user mode callbacks, that
%		may directly or indirectly lead to the affected affected routine being called. By default,
%		only a small number of system-provided routines is affected by such callbacks (such as
%		KiUserExceptionDispatcher \cite{Nebbett00}). Although unlikely in practice, the presence of third-party 
%		drivers may entail the possibility of other user mode routines to be called from
%		the kernel.
%		
%		The exampled provided are non-exhaustive.

%	\item Hardware Induced Events (excluding interrupts):
		
\section{Evaluation of Safety Concerns}
Having discussed prevalent challenges of runtime code modification, a number of conclusions may
be drawn. An important observation is that due to the inherent complexities and challenges
coming along with replacing multiple instructions, an approach that only requires replacing
a single instruction of non-quiescent code is clearly favorable.

Moreover, not all issues are equally critical. Some events such as concurring DMA operations
on memory contag the code to be modified are unlikely and uncommon enough to be neglected
in practice.

	  %UM: threads in same address space + processes sharing pages
	  %COW
	  %KM: all threads
	  %WriteProcMem? kernel grabbing into user space
	  %DMA?
	  
	  
	  




