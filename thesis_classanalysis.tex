\section{Concluding Remarks}
%\section{Analysis of Dynamic Instrumentation Techniques}
The tracing solutions presented as part of the discussion of 
tracing techniques vary in both their implementation as well as in the individual aims they pursue.
A general assessment is thus not easily possible. Yet, moving the focus
to function boundary tracing in the context of the Windows NT kernel and the applicability
of the techniques in this regard, allows a finer analysis of the strengths and weaknesses 
of these approaches to be made.

Solutions \emph{modifying the environment} generally require the least effort
to implement and avoid the challenges associated with code modification. 
Notwithstanding their lightweightness and robustness,
the applicability of these approaches is limited. The NT kernel includes 
several areas of potential applications \cite{Skywing07} where extensive 
use of function pointers is made, yet the share of functions traceable 
with this approach is, although hardly quantifiable, rather low. 

In sharp contrast to this, solutions \emph{interposing code execution}
are extremely versatile and universally applicable. The fact that
these solutions allow instrumentation on a much finer level than required
for function boundary tracing should also be considered positive. As
indicated before, however, the achilles heel of these solutions is
the bootstrapping. To successfully interpose code execution, the
tracing facility must either be present from the start of the respective
program on or it must be injected and be provided some form of jump aid. 
However, requiring the tracing facility to be present during program startup
has to be considered a contradiction to the tracing solution being truly \emph{dynamic}. In
the second case, the tracing solution can only be as good as its jump
start facility is -- the more limited the number of potential entry points
is, the more limited will the applicability of such a solution be.

% Perf?
% entry points?

Solutions relying on \emph{injecting and handling traps} or \emph{editing code} 
share many properties. They do not reach the versatility
of interposition solutions but compensate this shortcoming by usually being
less intrusive and providing a larger number of potential instrumentation/entry points.
Such solutions are, however, exposed to the challenges of dynamic 
code modification as discussed in section 
\ref{sec:ChallengesOfRuntimeCodeModification}. Addressing all of these
problems is vital for providing a code modifying solution that can be considered safe
enough for productive use.

The main advantage of avoiding traps and to favor code editing
techniques is performance. Trap handling introduces a significant
overhead which has to be paid for each event. The cost
of issuing and handling breakpoints has been reported to be even 
higher in case of the operating system running in a hypervisor such as 
Xen \cite{Hiramatsu07}. 

% trap --> raises irql?!

% hardware?

% freeze_processes on NT? --> multi-instr patching inherently unsafe?


% low instrusiveness as criteria --> use against virt/interp.. 



%\section{Relation to Dynamic Updating Solutions}
%So far, only tracing techniques and solutions have been discussed. However, certain aspects, 
%such as the potential necessity to modify code in order to adapt and intercept execution, are shared with another 
%type of software, namely dynamic updating solutions.
%
%The classic way of updating software involves replacing existing modules with new,
%updated copies and restarting the system. Depending on the individual system, 
%the shutdown-update-restart cycle can introduce a significant downtime, in particular
%when the system to be updated is the operating system itself.
%By providing a means to apply certain software updates to a running system, dynamic
%updating solutions promise to minimize downtime caused by software updates.
%
%A detailed analysis of the similarities and differences between dynamic tracing and 
%updating techniques is out of scope of this work. However, it is notable that several 
%updating solutions can be identified to use the technique of \emph{Editing Code}.
%
%Ksplice \cite{Ksplice08}, Dynamos \cite{Dynamos07} as well as Microsoft Hotpatching (discussed in 
%section \ref{sec:MsHotpatching}) are examples for dynamic updating solutions that 
%rely on code editing techniques to redirect execution flow from old, flawed routines 
%to corresponding new, updated routines. Although differing in detail, all three solutions 
%implement this redirection by injecting jumps to trampolines into the original code 
%and having the trampolines redirect execution to the new routine.
%


% Tracing: no relocation req'd
% No new code generation
% preservation of old routine

